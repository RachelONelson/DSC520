% Options for packages loaded elsewhere
\PassOptionsToPackage{unicode}{hyperref}
\PassOptionsToPackage{hyphens}{url}
%
\documentclass[
]{article}
\usepackage{lmodern}
\usepackage{amssymb,amsmath}
\usepackage{ifxetex,ifluatex}
\ifnum 0\ifxetex 1\fi\ifluatex 1\fi=0 % if pdftex
  \usepackage[T1]{fontenc}
  \usepackage[utf8]{inputenc}
  \usepackage{textcomp} % provide euro and other symbols
\else % if luatex or xetex
  \usepackage{unicode-math}
  \defaultfontfeatures{Scale=MatchLowercase}
  \defaultfontfeatures[\rmfamily]{Ligatures=TeX,Scale=1}
\fi
% Use upquote if available, for straight quotes in verbatim environments
\IfFileExists{upquote.sty}{\usepackage{upquote}}{}
\IfFileExists{microtype.sty}{% use microtype if available
  \usepackage[]{microtype}
  \UseMicrotypeSet[protrusion]{basicmath} % disable protrusion for tt fonts
}{}
\makeatletter
\@ifundefined{KOMAClassName}{% if non-KOMA class
  \IfFileExists{parskip.sty}{%
    \usepackage{parskip}
  }{% else
    \setlength{\parindent}{0pt}
    \setlength{\parskip}{6pt plus 2pt minus 1pt}}
}{% if KOMA class
  \KOMAoptions{parskip=half}}
\makeatother
\usepackage{xcolor}
\IfFileExists{xurl.sty}{\usepackage{xurl}}{} % add URL line breaks if available
\IfFileExists{bookmark.sty}{\usepackage{bookmark}}{\usepackage{hyperref}}
\hypersetup{
  pdftitle={assignment-2.1-NelsonRachel.R},
  pdfauthor={Rachel},
  hidelinks,
  pdfcreator={LaTeX via pandoc}}
\urlstyle{same} % disable monospaced font for URLs
\usepackage[margin=1in]{geometry}
\usepackage{color}
\usepackage{fancyvrb}
\newcommand{\VerbBar}{|}
\newcommand{\VERB}{\Verb[commandchars=\\\{\}]}
\DefineVerbatimEnvironment{Highlighting}{Verbatim}{commandchars=\\\{\}}
% Add ',fontsize=\small' for more characters per line
\usepackage{framed}
\definecolor{shadecolor}{RGB}{248,248,248}
\newenvironment{Shaded}{\begin{snugshade}}{\end{snugshade}}
\newcommand{\AlertTok}[1]{\textcolor[rgb]{0.94,0.16,0.16}{#1}}
\newcommand{\AnnotationTok}[1]{\textcolor[rgb]{0.56,0.35,0.01}{\textbf{\textit{#1}}}}
\newcommand{\AttributeTok}[1]{\textcolor[rgb]{0.77,0.63,0.00}{#1}}
\newcommand{\BaseNTok}[1]{\textcolor[rgb]{0.00,0.00,0.81}{#1}}
\newcommand{\BuiltInTok}[1]{#1}
\newcommand{\CharTok}[1]{\textcolor[rgb]{0.31,0.60,0.02}{#1}}
\newcommand{\CommentTok}[1]{\textcolor[rgb]{0.56,0.35,0.01}{\textit{#1}}}
\newcommand{\CommentVarTok}[1]{\textcolor[rgb]{0.56,0.35,0.01}{\textbf{\textit{#1}}}}
\newcommand{\ConstantTok}[1]{\textcolor[rgb]{0.00,0.00,0.00}{#1}}
\newcommand{\ControlFlowTok}[1]{\textcolor[rgb]{0.13,0.29,0.53}{\textbf{#1}}}
\newcommand{\DataTypeTok}[1]{\textcolor[rgb]{0.13,0.29,0.53}{#1}}
\newcommand{\DecValTok}[1]{\textcolor[rgb]{0.00,0.00,0.81}{#1}}
\newcommand{\DocumentationTok}[1]{\textcolor[rgb]{0.56,0.35,0.01}{\textbf{\textit{#1}}}}
\newcommand{\ErrorTok}[1]{\textcolor[rgb]{0.64,0.00,0.00}{\textbf{#1}}}
\newcommand{\ExtensionTok}[1]{#1}
\newcommand{\FloatTok}[1]{\textcolor[rgb]{0.00,0.00,0.81}{#1}}
\newcommand{\FunctionTok}[1]{\textcolor[rgb]{0.00,0.00,0.00}{#1}}
\newcommand{\ImportTok}[1]{#1}
\newcommand{\InformationTok}[1]{\textcolor[rgb]{0.56,0.35,0.01}{\textbf{\textit{#1}}}}
\newcommand{\KeywordTok}[1]{\textcolor[rgb]{0.13,0.29,0.53}{\textbf{#1}}}
\newcommand{\NormalTok}[1]{#1}
\newcommand{\OperatorTok}[1]{\textcolor[rgb]{0.81,0.36,0.00}{\textbf{#1}}}
\newcommand{\OtherTok}[1]{\textcolor[rgb]{0.56,0.35,0.01}{#1}}
\newcommand{\PreprocessorTok}[1]{\textcolor[rgb]{0.56,0.35,0.01}{\textit{#1}}}
\newcommand{\RegionMarkerTok}[1]{#1}
\newcommand{\SpecialCharTok}[1]{\textcolor[rgb]{0.00,0.00,0.00}{#1}}
\newcommand{\SpecialStringTok}[1]{\textcolor[rgb]{0.31,0.60,0.02}{#1}}
\newcommand{\StringTok}[1]{\textcolor[rgb]{0.31,0.60,0.02}{#1}}
\newcommand{\VariableTok}[1]{\textcolor[rgb]{0.00,0.00,0.00}{#1}}
\newcommand{\VerbatimStringTok}[1]{\textcolor[rgb]{0.31,0.60,0.02}{#1}}
\newcommand{\WarningTok}[1]{\textcolor[rgb]{0.56,0.35,0.01}{\textbf{\textit{#1}}}}
\usepackage{graphicx,grffile}
\makeatletter
\def\maxwidth{\ifdim\Gin@nat@width>\linewidth\linewidth\else\Gin@nat@width\fi}
\def\maxheight{\ifdim\Gin@nat@height>\textheight\textheight\else\Gin@nat@height\fi}
\makeatother
% Scale images if necessary, so that they will not overflow the page
% margins by default, and it is still possible to overwrite the defaults
% using explicit options in \includegraphics[width, height, ...]{}
\setkeys{Gin}{width=\maxwidth,height=\maxheight,keepaspectratio}
% Set default figure placement to htbp
\makeatletter
\def\fps@figure{htbp}
\makeatother
\setlength{\emergencystretch}{3em} % prevent overfull lines
\providecommand{\tightlist}{%
  \setlength{\itemsep}{0pt}\setlength{\parskip}{0pt}}
\setcounter{secnumdepth}{-\maxdimen} % remove section numbering

\title{assignment-2.1-NelsonRachel.R}
\author{Rachel}
\date{2020-06-13}

\begin{document}
\maketitle

\begin{Shaded}
\begin{Highlighting}[]
\CommentTok{# Assignment: 2.1 Assignment: 2014 American Community Survey}
\CommentTok{# Name: Nelson, Rachel}
\CommentTok{# Date: 2020-06-13}


\CommentTok{# For this assignment, you will need to load and activate the ggplot2 package. (I urge you to do the DataCamp exercise first!). For this deliverable, you should provide the following:}
\KeywordTok{rm}\NormalTok{(}\DataTypeTok{list=}\KeywordTok{ls}\NormalTok{())}
\KeywordTok{library}\NormalTok{(ggplot2)}
\KeywordTok{library}\NormalTok{(pastecs)}
\KeywordTok{setwd}\NormalTok{(}\StringTok{"C:/Users/Rachel/Desktop/College/DSC520/dsc520-master"}\NormalTok{)}
\NormalTok{acs_df <-}\StringTok{ }\KeywordTok{read.csv}\NormalTok{(}\StringTok{"data/acs-14-1yr-s0201.csv"}\NormalTok{)}

\CommentTok{# 1. What are the elements in your data (including the categories and data types)?}
\KeywordTok{summary}\NormalTok{(acs_df)}
\end{Highlighting}
\end{Shaded}

\begin{verbatim}
##               Id           Id2                                 Geography  
##  0500000US01073:  1   Min.   : 1073   Alameda County, California    :  1  
##  0500000US04013:  1   1st Qu.:12082   Allegheny County, Pennsylvania:  1  
##  0500000US04019:  1   Median :26112   Anne Arundel County, Maryland :  1  
##  0500000US06001:  1   Mean   :26833   Arapahoe County, Colorado     :  1  
##  0500000US06013:  1   3rd Qu.:39123   Baltimore city, Maryland      :  1  
##  0500000US06019:  1   Max.   :55079   Baltimore County, Maryland    :  1  
##  (Other)       :130                   (Other)                       :130  
##    PopGroupID      POPGROUP.display.label RacesReported         HSDegree    
##  Min.   :1    Total population:136        Min.   :  500292   Min.   :62.20  
##  1st Qu.:1                                1st Qu.:  631380   1st Qu.:85.50  
##  Median :1                                Median :  832708   Median :88.70  
##  Mean   :1                                Mean   : 1144401   Mean   :87.63  
##  3rd Qu.:1                                3rd Qu.: 1216862   3rd Qu.:90.75  
##  Max.   :1                                Max.   :10116705   Max.   :95.50  
##                                                                             
##    BachDegree   
##  Min.   :15.40  
##  1st Qu.:29.65  
##  Median :34.10  
##  Mean   :35.46  
##  3rd Qu.:42.08  
##  Max.   :60.30  
## 
\end{verbatim}

\begin{Shaded}
\begin{Highlighting}[]
\KeywordTok{attributes}\NormalTok{(acs_df)}
\end{Highlighting}
\end{Shaded}

\begin{verbatim}
## $names
## [1] "Id"                     "Id2"                    "Geography"             
## [4] "PopGroupID"             "POPGROUP.display.label" "RacesReported"         
## [7] "HSDegree"               "BachDegree"            
## 
## $class
## [1] "data.frame"
## 
## $row.names
##   [1]   1   2   3   4   5   6   7   8   9  10  11  12  13  14  15  16  17  18
##  [19]  19  20  21  22  23  24  25  26  27  28  29  30  31  32  33  34  35  36
##  [37]  37  38  39  40  41  42  43  44  45  46  47  48  49  50  51  52  53  54
##  [55]  55  56  57  58  59  60  61  62  63  64  65  66  67  68  69  70  71  72
##  [73]  73  74  75  76  77  78  79  80  81  82  83  84  85  86  87  88  89  90
##  [91]  91  92  93  94  95  96  97  98  99 100 101 102 103 104 105 106 107 108
## [109] 109 110 111 112 113 114 115 116 117 118 119 120 121 122 123 124 125 126
## [127] 127 128 129 130 131 132 133 134 135 136
\end{verbatim}

\begin{Shaded}
\begin{Highlighting}[]
\CommentTok{# The elements include ID=factor, ID2=int, Geography=factor, PhotoGroupID=int, PopGroup=factor, Races Reported=int, HSDegree=num, BachDegree=nium}

\CommentTok{# 2. Please provide the output from the following functions: str(); nrow(); ncol()}
\KeywordTok{str}\NormalTok{(acs_df)}
\end{Highlighting}
\end{Shaded}

\begin{verbatim}
## 'data.frame':    136 obs. of  8 variables:
##  $ Id                    : Factor w/ 136 levels "0500000US01073",..: 1 2 3 4 5 6 7 8 9 10 ...
##  $ Id2                   : int  1073 4013 4019 6001 6013 6019 6029 6037 6059 6065 ...
##  $ Geography             : Factor w/ 136 levels "Alameda County, California",..: 56 70 98 1 20 43 62 68 92 106 ...
##  $ PopGroupID            : int  1 1 1 1 1 1 1 1 1 1 ...
##  $ POPGROUP.display.label: Factor w/ 1 level "Total population": 1 1 1 1 1 1 1 1 1 1 ...
##  $ RacesReported         : int  660793 4087191 1004516 1610921 1111339 965974 874589 10116705 3145515 2329271 ...
##  $ HSDegree              : num  89.1 86.8 88 86.9 88.8 73.6 74.5 77.5 84.6 80.6 ...
##  $ BachDegree            : num  30.5 30.2 30.8 42.8 39.7 19.7 15.4 30.3 38 20.7 ...
\end{verbatim}

\begin{Shaded}
\begin{Highlighting}[]
\KeywordTok{nrow}\NormalTok{(acs_df)}
\end{Highlighting}
\end{Shaded}

\begin{verbatim}
## [1] 136
\end{verbatim}

\begin{Shaded}
\begin{Highlighting}[]
\KeywordTok{ncol}\NormalTok{(acs_df)}
\end{Highlighting}
\end{Shaded}

\begin{verbatim}
## [1] 8
\end{verbatim}

\begin{Shaded}
\begin{Highlighting}[]
\CommentTok{# 3. Create a Histogram of the HSDegree variable using the ggplot2 package.}
\KeywordTok{ggplot}\NormalTok{(acs_df, }\KeywordTok{aes}\NormalTok{(HSDegree))}
\end{Highlighting}
\end{Shaded}

\includegraphics{assignment-2.1-NelsonRachel_files/figure-latex/unnamed-chunk-1-1.pdf}

\begin{Shaded}
\begin{Highlighting}[]
\CommentTok{#   a. Set a bin size for the Histogram.}
\KeywordTok{ggplot}\NormalTok{(acs_df, }\KeywordTok{aes}\NormalTok{(HSDegree)) }\OperatorTok{+}\StringTok{ }\KeywordTok{geom_histogram}\NormalTok{(}\DataTypeTok{bins =} \DecValTok{10}\NormalTok{)}
\end{Highlighting}
\end{Shaded}

\includegraphics{assignment-2.1-NelsonRachel_files/figure-latex/unnamed-chunk-1-2.pdf}

\begin{Shaded}
\begin{Highlighting}[]
\CommentTok{#   b. Include a Title and appropriate X/Y axis labels on your Histogram Plot.}
\KeywordTok{ggplot}\NormalTok{(acs_df, }\KeywordTok{aes}\NormalTok{(HSDegree)) }\OperatorTok{+}\StringTok{ }\KeywordTok{geom_histogram}\NormalTok{(}\DataTypeTok{bins =} \DecValTok{10}\NormalTok{) }\OperatorTok{+}\StringTok{ }\KeywordTok{ggtitle}\NormalTok{(}\StringTok{"Histogram of HS Degree"}\NormalTok{) }\OperatorTok{+}\StringTok{ }\KeywordTok{xlab}\NormalTok{(}\StringTok{"HSDegree"}\NormalTok{) }\OperatorTok{+}\StringTok{ }\KeywordTok{ylab}\NormalTok{(}\StringTok{"Count of occurrance"}\NormalTok{) }
\end{Highlighting}
\end{Shaded}

\includegraphics{assignment-2.1-NelsonRachel_files/figure-latex/unnamed-chunk-1-3.pdf}

\begin{Shaded}
\begin{Highlighting}[]
\CommentTok{# 4. Answer the following questions based on the Histogram produced:}
\CommentTok{#   a. Based on what you see in this histogram, is the data distribution unimodal?}
\CommentTok{#       answer: Yes, the distribution is unimodal as it has one clear peak}
\CommentTok{#   b. Is it approximately symmetrical?}
\CommentTok{#       answer: No, the data is skewed to the left}
\CommentTok{#   c. Is it approximately bell-shaped?}
\CommentTok{#       answer:   No}
\CommentTok{#   d. Is it approximately normal?}
\KeywordTok{shapiro.test}\NormalTok{(acs_df}\OperatorTok{$}\NormalTok{HSDegree) }\CommentTok{#I'm using the sharpiro test to check if the data is normal}
\end{Highlighting}
\end{Shaded}

\begin{verbatim}
## 
##  Shapiro-Wilk normality test
## 
## data:  acs_df$HSDegree
## W = 0.87736, p-value = 3.194e-09
\end{verbatim}

\begin{Shaded}
\begin{Highlighting}[]
\CommentTok{#       answer: No, the p-value < .05 p-value which implies the distribution of the data is significantly different from }
\CommentTok{#       normal distribution.}
\CommentTok{#   e. If not normal, is the distribution skewed? If so, in which direction?}
\CommentTok{#       answer: Yes, the data is skewed to the left}
\CommentTok{#   f. Include a normal curve to the Histogram that you plotted.}
\CommentTok{#       answer: see below}
\NormalTok{gg <-}\StringTok{ }\KeywordTok{ggplot}\NormalTok{(acs_df, }\KeywordTok{aes}\NormalTok{(}\DataTypeTok{x=}\NormalTok{HSDegree))}
\NormalTok{gg <-}\StringTok{ }\NormalTok{gg }\OperatorTok{+}\StringTok{ }\KeywordTok{geom_histogram}\NormalTok{(}\DataTypeTok{binwidth=}\DecValTok{2}\NormalTok{, }\DataTypeTok{colour=}\StringTok{"black"}\NormalTok{, }
                          \KeywordTok{aes}\NormalTok{(}\DataTypeTok{y=}\NormalTok{..density.., }\DataTypeTok{fill=}\NormalTok{..count..))}
\NormalTok{gg <-}\StringTok{ }\NormalTok{gg }\OperatorTok{+}\StringTok{ }\KeywordTok{scale_fill_gradient}\NormalTok{(}\StringTok{"Count"}\NormalTok{, }\DataTypeTok{low=}\StringTok{"#DCDCDC"}\NormalTok{, }\DataTypeTok{high=}\StringTok{"#7C7C7C"}\NormalTok{)}
\NormalTok{gg <-}\StringTok{ }\NormalTok{gg }\OperatorTok{+}\StringTok{ }\KeywordTok{stat_function}\NormalTok{(}\DataTypeTok{fun=}\NormalTok{dnorm,}
                         \DataTypeTok{color=}\StringTok{"red"}\NormalTok{,}
                         \DataTypeTok{args=}\KeywordTok{list}\NormalTok{(}\DataTypeTok{mean=}\KeywordTok{mean}\NormalTok{(acs_df}\OperatorTok{$}\NormalTok{HSDegree), }
                                   \DataTypeTok{sd=}\KeywordTok{sd}\NormalTok{(acs_df}\OperatorTok{$}\NormalTok{HSDegree)))}
\NormalTok{gg}
\end{Highlighting}
\end{Shaded}

\includegraphics{assignment-2.1-NelsonRachel_files/figure-latex/unnamed-chunk-1-4.pdf}

\begin{Shaded}
\begin{Highlighting}[]
\CommentTok{# g. Explain whether a normal distribution can accurately be used as a model for this data.}
\CommentTok{#       answer: No, because the data does not have the characteristics of normal data (99.9% of the data will not be within six sigma (or three standard deviations either way)) since the data is skewed left}

\CommentTok{# 5. Create a Probability Plot of the HSDegree variable.}
\KeywordTok{ggplot}\NormalTok{(acs_df, }\KeywordTok{aes}\NormalTok{(}\DataTypeTok{sample=}\NormalTok{HSDegree))}\OperatorTok{+}\KeywordTok{stat_qq}\NormalTok{()}
\end{Highlighting}
\end{Shaded}

\includegraphics{assignment-2.1-NelsonRachel_files/figure-latex/unnamed-chunk-1-5.pdf}

\begin{Shaded}
\begin{Highlighting}[]
\CommentTok{# 6. Answer the following questions based on the Probability Plot:}
\CommentTok{#   a. Based on what you see in this probability plot, is the distribution approximately normal? Explain how you know.}
\CommentTok{#       answer: No, if the data was normal the line would be linear. Because of the curvature, you can tell that the data is nor normal.}
\CommentTok{#   b. If not normal, is the distribution skewed? If so, in which direction? Explain how you know.}
\CommentTok{#       answer: Yes, the distribution is skewed to the left. You can tell by the downward curvature of the line. }

\CommentTok{# 7. Now that you have looked at this data visually for normality, you will now quantify normality with numbers using the stat.desc() function. Include a screen capture of the results produced.}
\KeywordTok{stat.desc}\NormalTok{(acs_df,  }\DataTypeTok{norm =} \OtherTok{FALSE}\NormalTok{) }
\end{Highlighting}
\end{Shaded}

\begin{verbatim}
##          Id          Id2 Geography PopGroupID POPGROUP.display.label
## nbr.val  NA 1.360000e+02        NA        136                     NA
## nbr.null NA 0.000000e+00        NA          0                     NA
## nbr.na   NA 0.000000e+00        NA          0                     NA
## min      NA 1.073000e+03        NA          1                     NA
## max      NA 5.507900e+04        NA          1                     NA
## range    NA 5.400600e+04        NA          0                     NA
## sum      NA 3.649306e+06        NA        136                     NA
## median   NA 2.611200e+04        NA          1                     NA
## mean     NA 2.683313e+04        NA          1                     NA
## SE.mean  NA 1.323036e+03        NA          0                     NA
## CI.mean  NA 2.616557e+03        NA          0                     NA
## var      NA 2.380576e+08        NA          0                     NA
## std.dev  NA 1.542911e+04        NA          0                     NA
## coef.var NA 5.750024e-01        NA          0                     NA
##          RacesReported     HSDegree   BachDegree
## nbr.val   1.360000e+02 1.360000e+02  136.0000000
## nbr.null  0.000000e+00 0.000000e+00    0.0000000
## nbr.na    0.000000e+00 0.000000e+00    0.0000000
## min       5.002920e+05 6.220000e+01   15.4000000
## max       1.011671e+07 9.550000e+01   60.3000000
## range     9.616413e+06 3.330000e+01   44.9000000
## sum       1.556385e+08 1.191800e+04 4822.7000000
## median    8.327075e+05 8.870000e+01   34.1000000
## mean      1.144401e+06 8.763235e+01   35.4610294
## SE.mean   9.351028e+04 4.388598e-01    0.8154527
## CI.mean   1.849346e+05 8.679296e-01    1.6127146
## var       1.189207e+12 2.619332e+01   90.4349886
## std.dev   1.090508e+06 5.117941e+00    9.5097313
## coef.var  9.529072e-01 5.840241e-02    0.2681741
\end{verbatim}

\begin{Shaded}
\begin{Highlighting}[]
\KeywordTok{stat.desc}\NormalTok{(acs_df}\OperatorTok{$}\StringTok{`}\DataTypeTok{HSDegree}\StringTok{`}\NormalTok{, }\DataTypeTok{basic =} \OtherTok{FALSE}\NormalTok{, }\DataTypeTok{norm =} \OtherTok{TRUE}\NormalTok{)}
\end{Highlighting}
\end{Shaded}

\begin{verbatim}
##        median          mean       SE.mean  CI.mean.0.95           var 
##  8.870000e+01  8.763235e+01  4.388598e-01  8.679296e-01  2.619332e+01 
##       std.dev      coef.var      skewness      skew.2SE      kurtosis 
##  5.117941e+00  5.840241e-02 -1.674767e+00 -4.030254e+00  4.352856e+00 
##      kurt.2SE    normtest.W    normtest.p 
##  5.273885e+00  8.773635e-01  3.193634e-09
\end{verbatim}

\begin{Shaded}
\begin{Highlighting}[]
\CommentTok{# 8. In several sentences provide an explanation of the result produced for skew, kurtosis, and z-scores. In addition, explain how a change in the sample size may change your explanation?}
\CommentTok{#       answer: Skew measures the asymmetry of the data and kurtosis measure the peakedness of the data. }
\CommentTok{#       For skew, a score of 0 represents a perfect normal distribution. In this case, the skew is -1.67, the negative number indicates the data is skewed to the left}
\CommentTok{#       For kutorsis, a value of 0 represents a perfect normal distribution. In this case, the kurtosis is 4.35, which indicates the peakedness of the data is not considered normal}
\CommentTok{#       The Z score For medium-sized samples, you can reject the null hypothesis at absolute z-value over 3.29,and determine the data is considered non-normal.}
\CommentTok{#       The more samples, the greater probability of rejecting that the values come from a normal distribution because it becoems more sensitive to small deviations within the data}
\end{Highlighting}
\end{Shaded}

\end{document}
